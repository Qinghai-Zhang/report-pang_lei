%%%%%%%%%%%%%%%%%%%%%%%%%%%%%%%%%%%%%%%%%
% fphw 
% LaTeX Template
% Version 1.0 (27/04/2019)
%
% This template originates from:
% https://www.LaTeXTemplates.com
%
% Authors:
% Class by Felipe Portales-Oliva (f.portales.oliva@gmail.com) with template 
% content and modifications by Vel (vel@LaTeXTemplates.com)
%
% Template (this file) License:
% CC BY-NC-SA 3.0 (http://creativecommons.org/licenses/by-nc-sa/3.0/)
%
%%%%%%%%%%%%%%%%%%%%%%%%%%%%%%%%%%%%%%%%%

%----------------------------------------------------------------------------------------
%	PACKAGES AND OTHER DOCUMENT CONFIGURATIONS
%----------------------------------------------------------------------------------------

\documentclass[
	12pt, % Default font size, values between 10pt-12pt are allowed
	%letterpaper, % Uncomment for US letter paper size
	%spanish, % Uncomment for Spanish
]{fphw}

% Template-specific packages
\usepackage[utf8]{inputenc} % Required for inputting international characters
\usepackage[T1]{fontenc} % Output font encoding for international characters
\usepackage{mathpazo} % Use the Palatino font

\usepackage{graphicx} % Required for including images

\usepackage{booktabs} % Required for better horizontal rules in tables

\usepackage{listings} % Required for insertion of code

\usepackage{enumerate} % To modify the enumerate environment

\usepackage[UTF8]{ctex} % 中文包

%----------------------------------------------------------------------------------------
%	ASSIGNMENT INFORMATION
%----------------------------------------------------------------------------------------

\title{月报\#02, 2023} % Assignment title

\author{庞雷} % Student name

\date{02 28, 2023} % Due date

\institute{浙江大学 \\ 数学科学学院} % Institute or school name

\professor{张庆海} % Professor or teacher 

\begin{document}
\maketitle % Output the title, created automatically using the information in the custom commands above

%----------------------------------------------------------------------------------------
%	CONTENT
%----------------------------------------------------------------------------------------

\section*{学习内容\uppercase\expandafter{\romannumeral1}}
精读《微分方程数值解》(一本绿皮的中文教材)全书,对这门课程有一个初步的认识.精读讲义7-9章,完成前半学期课程预习任务.
完成纠错若干,如讲义上的$Theorem\quad8.94\\(Perron-Frobenius)$表述错误.略读10-12章,将预习任务安排在3月进行
(p.s.精读表示逐行逐句阅读,不绕开定理的证明和式子的计算).


\section*{学习内容\uppercase\expandafter{\romannumeral2}}
精读《$Essential\quad C++$》全书,个人感觉该书内容偏简单,不能获取$C++$的高端知识或技巧,但其内容精炼,可以快速重温$C++$
的核心知识(或者作为入门书籍使用).除此之外,精读《$Effective\quad C++$》的前12项条款(前2章),学习$C++$更专业的知识,
将该书后续章节学习留置3,4月份进行,好书需要慢慢消化.


\section*{学习内容\uppercase\expandafter{\romannumeral3}}
精读《常微分方程》与《偏微分方程》两书,准备这学期的博士生资格考试(如果有的话).


\section*{学习内容\uppercase\expandafter{\romannumeral4}}
精读《$Graph\quad Theory\quad and\quad Complex\quad Networks$》前7章,讲义编写工作缓步进行中.


\section*{学习内容\uppercase\expandafter{\romannumeral5}}
给人工智能社团做了一次读书报告,分享《$Pattern\quad Recognition\quad and\quad Machine\quad Learning$》


\section*{学习内容\uppercase\expandafter{\romannumeral6}}
坚持每日观看45分钟人文历史方向的英文纪录片,学习学科外的知识.


\section*{学习内容\uppercase\expandafter{\romannumeral7}}
每周三次及以上健身房打卡,坚持体育锻炼.


\section*{下个月安排}
除去必要的上课,社交活动以及课程作业,计划精读讲义的10-12章节,完成课程后半部分的预习工作.系统学习numpy和matplotlib这两个库,
以便后期科研作图用.坚持编程,外语学习和体育锻炼.如有多余时间,学习实复变函数以及泛函分析. 


\end{document}
